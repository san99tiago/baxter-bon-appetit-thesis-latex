% THE AMAZING BAXTER FEEDING ROBOT 
\documentclass[11pt]{report} %Letter size, letter type and report article structure

\usepackage{geometry}
\geometry{
 a4paper,
 total={210mm,297mm},
 left=40mm,
 right=20mm,
 top=40mm,
 bottom=30mm,
 }
 
\usepackage[spanish, english]{babel} % Recognize Spanish language

\usepackage[utf8]{inputenc} % Recognize Latin characters

\usepackage{amsmath} % Insert math formulas/functions

\usepackage{amsfonts} % Extended set of fonts

\usepackage{graphicx} % Include graphics in document

\usepackage{siunitx} % Consistent set of units with rules on how they are to be used

\usepackage{caption} % Caption for images, formulas and tables

\usepackage{natbib} % Cite library (to reference in APA format)
\usepackage[breaklinks=true]{hyperref} % Handle cross-referencing commands
\urlstyle{same}
\usepackage{url}


\usepackage{comment} % Extra big-comments functionalities

\usepackage{textcomp} % For "R" and "TM" symbols

\usepackage{subcaption} % For sub-figures

\usepackage{array} % Usage of arrays

\usepackage{booktabs} % Better quality tables

\usepackage{longtable} % Allow tables to flow over page boundaries

\usepackage{makecell} % Extra functionalities to table cells tabs

\usepackage{multirow} % Extra functionalities to table cells row/cols

\usepackage{float} % Improve figures/tables position in page

\usepackage{algorithm} % Write algorithms easily

\usepackage[noend]{algpseudocode} % Improve pseudo-code algorithms

% Extra functionalities for vertical text
\newcommand*\rot{\rotatebox{90}}
\newcommand*\OK{\ding{51}}

% Reestablish specific low-level configurations
\renewcommand\theadalign{cb}
\renewcommand\theadfont{\bfseries}
\renewcommand\theadgape{\Gape[4pt]}
\renewcommand\cellgape{\Gape[4pt]}

% Remove first-indent in all document
\setlength\parindent{0pt}

% Avoid annoying "under full hbox oversize comment"
\hbadness = 2147483647
\vbadness = 2147483647

% Extra math operators values
\DeclareMathOperator{\T}{T}
\DeclareMathOperator{\g}{g}
\DeclareMathOperator{\f}{f}
\DeclareMathOperator{\F}{F}
\DeclareMathOperator{\h}{p}
\DeclareMathOperator{\p}{p}
\DeclareMathOperator{\U}{U}

% ---------------------------------------------------------

\begin{document}

% ---------------------------------------------------------

\begin{titlepage}

\begin{center}

\begin{Large}

\vspace*{1cm}
\textbf{ASSISTED ROBOTICS FOR FEEDING INDIVIDUALS WITH UPPER LIMB DISABILITIES}\\[6.5cm]

\textbf{Santiago García Arango} \\
\textbf{Elkin Javier Guerra Galeano}

\end{Large}

\vfill

\includegraphics[width=6cm]{assets/imgs/logoeia.png}\\

\begin{large}
\textbf{EIA UNIVERSITY}\\
\textbf{MECHATRONICS ENGINEERING}\\
\textbf{ENVIGADO, COLOMBIA}\\
\textbf{2021}\\
\end{large}
\end{center}
\end{titlepage}

% ---------------------------------------------------------

\begin{titlepage}
\begin{center}

\begin{Large}

\vspace*{1cm}
\textbf{ASSISTED ROBOTICS FOR FEEDING INDIVIDUALS WITH UPPER LIMB DISABILITIES}\\[3.5cm]

\textbf{Santiago García Arango}\\
\textbf{Elkin Javier Guerra Galeano}\\[2cm]

\end{Large}

\begin{large}

\textbf{Final project to obtain the degree of:}\\
\textbf{Mechatronics Engineer}\\[2cm]

\textbf{Director of the project:}\\
\textbf{Dolly Tatiana Manrique Espíndola, M.Sc., Ph.D.}\\[1cm]

\end{large}

\vfill

\includegraphics[width=5cm]{assets/imgs/logoeia.png}\\

\begin{large}
\textbf{EIA UNIVERSITY}\\
\textbf{MECHATRONICS ENGINEERING}\\
\textbf{ENVIGADO, COLOMBIA}\\
\textbf{2021}\\
\end{large}

\end{center}
\end{titlepage}

% ---------------------------------------------------------

\begin{titlepage}
\begin{center}

\begin{Large}

\vspace*{1cm}
\textbf{ACKNOWLEDGMENTS}\\[3.5cm]

\textbf{Acknowledgments 1}\\

\end{Large}

\begin{Large}

\vspace*{3cm}

\textbf{Acknowledgments 2}\\

\end{Large}

\vfill

\end{center}
\end{titlepage}

% ---------------------------------------------------------

\tableofcontents

% ---------------------------------------------------------

\cleardoublepage
\listoffigures

% ---------------------------------------------------------

\cleardoublepage
\listoftables

% ---------------------------------------------------------

\cleardoublepage
\chapter*{List of Symbols}

\begin{longtable}{p{5mm} c p{120mm} }

\multicolumn{3}{l}{\textbf{Position Variables}}\\
\\
$\zeta$ & --- & Vector of something...\\
\\
\multicolumn{3}{l}{\textbf{Robot's parameters}}\\
\\
$P$ & --- & Centroid of ...\\

\\
\multicolumn{3}{l}{\textbf{Control Theory parameters}}\\
\\
$t$ & --- & Time ...\\

\\
\multicolumn{3}{l}{\textbf{Probability functions}}\\
\\
$\g(.)$ & --- & Probability function for ...\\

\\


\end{longtable}

% ---------------------------------------------------------

\chapter*{Abstract}

\pagenumbering{arabic} 

The exponential growth of technology has made it possible to achieve new solutions to improve the human beings' quality of life. One of the areas that has experienced the greatest development in the last 20 years is robotics and its derivatives. Currently, there has been a significant increase in the number of people with motor disabilities in the upper limbs, including more than 10 million people with Parkinson's disease and a number of individuals who, due to other circumstances, have lost the mobility of their upper limbs. This group of people not only have major difficulties in the daily task of feeding, but can also experience severe problems of malnutrition and loss of self-esteem. This is why, in this project, an exploratory research will be conducted focused on the development of an active robotic solution, using Baxter Robot, which can give support in the feeding process for these individuals and, at the same time, has conditions of improvement compared to the robotic alternatives that exist in the current market.\\

This development will seek a positive impact for all individuals who fit within the exposed problem and a design methodology will be carried out, oriented to the search for a scalable solution, with the ability to recognize the position of the mouth of individuals through Computer Vision algorithms and with the advantage of being Open Source. It is expected that, at the end of this research, relevant advances will be generated in the development of active robotic solutions for the assistance of this population and the scientific knowledge of robotics in Colombia and the world.\\


\textbf{Keywords:} Robotics, Computer Vision, Control Theory, Denavit Hartenberg, Algorithms, Software, Anomalies, Active Feeding, Autonomous System.

% ---------------------------------------------------------

\chapter*{Introduction}

The overall population that have upper limb special needs or suffer from motor disabilities are more likely to have problems of malnutrition, a reduction in the performance in the activities of daily living, and loss of self-esteem  \citep{cite_ICBF_technical_article, cite_upper_limb_disabilities_self_steem}.\\

In this research project, we will dive into the design of a possible solution for these problems, using Baxter Robot as an active feeding solution that enables individuals to accomplish of the most important daily life activities: eating.\\

The project covers some important stages for the complete design: \\

\begin{enumerate}
    \item General understanding of the overall design based on human-being needs.
    \item Specific planning and implementation of each sub-system of the complete design.
    \item Practical experiments to validate the functionalities of the designs.
    \item Final results analysis and next steps for future improvements.
\end{enumerate}

Throughout this article, we will be exposing the theoretical and practical steps that were involved in the design of the Baxter Feeding Robot solution.\\

At the end of the article, we will expose the most important ethical and real life considerations of designing a robotics solution that interacts with human beings.

% ---------------------------------------------------------

\chapter{Initial Steps}

\section{Understanding the problem}

\subsection{Problem's context and analysis}

According to the Technical Guide of the Food and Nutrition Component for the Population with Disabilities of the Colombian Institute of Family Welfare (ICBF), this group of individuals are more likely to have nutritional problems \citep{cite_ICBF_technical_article}. Likewise, individuals with upper limb disabilities, have a significant reduction in activities of daily living (especially feeding) and may become impaired in their psychological state \citep{cite_upper_limb_disabilities_self_steem}.\\

It is estimated that there are more than 10 million people with Parkinson worldwide, of which a significant percentage have critical mobility problems in their upper extremities \citep{cite_parkinson_foundation_total_cases}. The incidence of people with Parkinson's disability increases with the age of the individuals, but there is an approximate 4\% of them who are diagnosed before the age of 50 years. According to studies conducted by the Parkinson Foundation (PF), men are more likely than women to have this disease. Another factor that is of relevance to this problem is that there is currently a higher prevalence of pain and disabilities in upper limbs in young university populations, which implies an increase in the group of people who will have joint disabilities in the future and will be more likely to present malnutrition problems due to the difficulties of performing the task of feeding by their own hands \citep{cite_park_active_robot_assisted_feeding}.\\

This is why it is necessary to look for technological solutions that allow this group of individuals to receive support and assistance in daily tasks, especially feeding. Considering this context, a branch of robotics known as "Assisted Robotics" is of interest, where robotic systems seek to support individuals with disabilities to perform daily tasks \citep{cite_assited_robotics_stanford_lecture_jaffe}.\\

Advances in assisted robotics have led to the development of various systems that seek to assist people with disabilities in the task of feeding (scooping). There are commercial robotic platforms, which seek to fulfill the need to feed patients with upper limb motor disabilities, but these have a number of limitations that restrict and limit their ideal performance. The best known commercial robotic systems for these tasks, such as Myspoon, Bestic Arm, Meal Buddy, Mealtime and Obi, are passive in nature. This means that they do not have the ability to adapt to the dynamic conditions of the user's mouth position, nor the ability to detect anomalies in the feeding process \citep{cite_park_active_robot_assisted_feeding}.\\

A restriction of great relevance found in commercial robotic systems is that they are not able to adapt their behavior to temporary changes in the position of the user's mouth, because their system is passive, limiting the ability to know this information that is required for a correct feeding action. Another limitation of these systems is that they fail to identify anomalies or strange behaviors in the feeding process, generating additional risks for the patient in case of any emergency or eventuality. Finally, these systems are implemented with restricted code, limiting their modifications in the source code and restricting the possibility of replicating these algorithms for the entire population that may need them.\\

Taking into account this worldwide problem, with the aim of improving the quality of life for individuals who have motor disability problems in upper limbs, it is important and of great relevance to seek an active robotic solution that can support the feeding process of these individuals, with a number of significant improvements over current commercial robotic systems. This solution should be able to adapt its dynamic movements constantly according to the position of the user's mouth, be aware of the environment, check for possible anomalies that may occur in it, be accessible and affordable for the population affected by their motor disabilities, be scalable to any other similar robotic system that is programmable and, similarly, should promote and seek innovation and the technological development of related robotic systems in Colombia and the world.\\

Based on the previous arguments, this research will aim to solve the question of:
How to implement an active robotic system for feeding (scooping) patients with motor disabilities in upper extremities, using the Baxter cobot of the EIA University?\\

In order to find a solution for the lack of active commercial robotic systems for feeding individuals, that also have the ability to adapt their movements according to the position of the user's mouth and the conditions of the environment, based on Mechatronics Engineering, several technological solutions can be proposed to integrate the main branches of electronics, mechanics, control systems and software development, to find an optimal solution to this identified problem \citep{cite_university_eia_general}.\\

This is why it is being proposed to develop a programmed robotic solution, which has an user interface activated by voice commands or friendly buttons, to generate a reliable and safe solution for the needs of people with motor disabilities in the upper limbs, to perform the daily activity of feeding. This solution, being open source, can be easily extrapolated to any programmable robotic platform, generating an additional step in the development of assisted robotic systems.\\

Considering the concept of assisted robotics and the importance of helping these individuals, the answer to provide a viable, scalable and safe solution to this problem is the usage of collaborative robots under the approach of working with humans  \citep{cite_rethink_robotics_baxter_factory_worker}. This can be achieved through various robotic alternatives, but it was decided to use Baxter robot from the company Rethink Robotics, which was developed under the concept of "Cobot", that means, a collaborative robot to work together with humans and is available in the laboratories of the university campus \citep{cite_university_eia_general}.\\

Due to the arguments exposed above, we want to look for a solution with the Baxter cobot, using the internal architecture of the software components integrated with the infrastructure offered by this company. This solution must be able to integrate the development of the software architecture, computational algorithms, kinematic models, video processing with computer vision, user interface,  and the necessary connections of these components for the correct implementation of the robotic solution, which will be focused on improving the quality of life of individuals who will benefit from this technology. Likewise, the system will represent a series of elements scalable to any other robotic platform with similar operating conditions and will seek to generate a positive impact on the development of assisted robotic systems for the community.\\

The development of a solution to this problem has multiple benefits, both for individuals with motor disabilities in the upper limbs, as well as for the scientific and medical community in Colombia. That is why, by developing this project, we want to take an additional step in the research and applications of assisted robotics, in order to seek to improve the quality of life of people with motor disabilities in the upper limbs.\\

This contribution can allow a person with these conditions, to perform the essential activity of feeding, without the need of having an external person performing this task. In the same way, this generates an increase in self-esteem and quality of life for these individuals.
At the same time, a key factor of the project is that it will positively contribute to the pursuit of three of the global objectives of sustainable development, especially: "Health and well-being", "Industry, innovation and infrastructure" and "Reduction of inequalities" \citep{cite_united_nations_sustainable_development}.\\

In addition to the previous reasons, one of the most important motives for carrying out this project is that there is a large gap between the robotics of the leading countries in technology and Colombian robotics. This is why the project will provide a methodological approach for robotics research and a technological progress that can be replicated in any Colombian institution that has access to robotic platforms, generating a community that aims to improve robotic developments at a national level.\\


% ---------------------------------------------------------

\newpage

\section{Research objectives}

In this chapter, we will dive into the objectives of the research that are expected by the end of the project.\\

It is of high importance to have a general and specific planning of the objectives to be developed in the evolution of the project. The following are the proposed objectives for the final scope of the solution proposed in this degree project.\\

\subsection{General objective}

Implement an active feeding system (scooping) for patients with motor disabilities in upper extremities, using the Baxter cobot of EIA University.

\subsection{Specific objectives}

\begin{enumerate}

\item Generate the smooth paths and trajectories between the position of the tool and the user's mouth, through a face detection system, which returns the absolute coordinates of the user's mouth based on the OpenCV library.

\item Implement a control strategy for tracking the proposed paths and trajectories, through a kinematic processing, to generate the action commands to move the joints of the Baxter cobot and validate its performance with a control algorithm, using the current position of the user's mouth as direct feedback to the system.

\item Design an algorithm for anomaly or error detection in the feeding process, which provides an action signal to stop the system in case of unexpected events.

\item Integrate the sub-systems of path and trajectory generation, control strategy and anomaly detection, using the Robot Operating System (ROS) libraries and tools.

\item Develop an user interface, which enables to send the commands to start the feeding and, if necessary, to stop the process.

\end{enumerate}

% ---------------------------------------------------------

\newpage

\section{Reference framework}

\subsection{Historical background}

Robotics was initially conceived as one of the branches of science with the greatest impact on industrial automation and processes that allow repetitive tasks to be performed. However, the concept of robotics has been expanding its scope and applications, because there are new sectors and scenarios where robots with new objectives and capabilities are becoming more frequent. Some of these scenarios are focused on improving the quality of life of human beings. Two major current scenarios of robotics are: assisted robotics and collaborative robotics (Decker, Fischer, & Ott, 2017).\\

Assisted robotics refers to all types of robots that have the ability to collect information from the environment, process that information, and perform tasks or actions that benefit individuals with any type of disability (Jaffe D. L., 2012). Based on this definition, it can be concluded that assisted robotics enables the improvement of the quality of life of people with any disability, including the human beings with motor impairments in the upper limbs.\\

The existing studies and researches related to robotics focused on patient feeding have had great advances in the last 10 years. Furthermore , it can be visualized that most current works are directed to the search and improvement of systems for feeding individuals in an active way, i.e., systems that autonomously feed patients (Park, et al., 2020).\\

In order to effectively structure and summarize some of the researches, degree papers and patents related to the topic of assisted robotics, a literature review of the last few years was carried out to collect the objectives, methodologies, solutions, strategies and conclusions of these projects. Some of the most relevant ones are presented below:\\

In the article "Active robot-assisted feeding with a general-purpose mobile manipulator: Design, evaluation, and lessons learned" (Park, et al., 2020), published in the scientific journal "Robotics and Autonomous Systems 124", an overview of robotic systems focused on patient feeding is presented, providing a contrast of commercial platforms and those under research and development. In this study, an algorithm and software architecture based on a PR2 robotic manipulator was implemented to achieve autonomous feeding of patients with upper extremity disabilities. The characteristics, methodologies and solutions found in similar projects were presented, making a contrast between them and exposing new proposals that made use of some of the knowledge found by all previous investigations. Besides, the article reflects an exhaustive work with a good overview of previous knowledge to develop similar robotic solutions in the future, as it not only exposes the development of their research in the particular context, but also proposes the most important elements to take into account in assisted robotics projects.\\

Another study of major relevance to the state of the art of assisted robotics is "SAM, an Assistive Robotic Device Dedicated to Helping Persons with Quadriplegia: Usability Study" (Fattal, et al., 2017). In this research, important concepts about assisted robotics and the current delimitation of various robotic applications in this area of knowledge are exposed. Likewise, the authors expose the importance of defining the factors for the success of robotic tasks, since the evaluation of the viability of implementation, is linked to correctly defining the success of the corresponding subroutines. Likewise, several interfaces that can be implemented for quadriplegic users are shown and their advantages and disadvantages are contrasted.\\

Additionally, by researching related studies, a very important element was identified in the development of a robotic system that interacts with individuals: the detection of anomalies. To give context to this feature, two research articles were analyzed, which were "A Multimodal Execution Monitor with Anomaly Classification for Robot-Assisted Feeding" (Park, et al, 2017) and "A Multimodal Anomaly Detector for Robot-Assisted Feeding Using an LSTM-based Variational Autoencoder" (Park, Hoshi, & Kemp, 2018). From these scientific papers, we gained a deeper insight into more than 5 mathematical methodologies for strategically detecting anomalies. Similarly, they expose analytical procedures that are of great importance in the structure of algorithms that focus on searching for anomaly patterns, enabling us to have these as a starting point to develop new strategies for future robotic implementations.\\

Regarding patents, the United States patent called "Mobile human-friendly assistive robot" (United States Patent No. US 201700.95382A1, 2017) was mainly reviewed. In this publication, it was possible to identify a viable structure for the communications architecture developed in assisted robotics projects, enabling the selection of the principal and secondary components in the infrastructure to be developed for robotic projects.\\


A very interesting article for the understanding of assisted robotics projects is "Robots for humanity: Using collaborative robots to help people with disabilities" (Chen, et al., 2013). In this document, it is exposed the relevance of simplifying the graphical user interface, through the use of mathematical tools such as elliptic coordinates and an interface compatible with sound commands. Similarly, the authors propose a strategic error detection system by translating the forces implemented by the robotic system in different mathematical approaches.\\


\subsection{Theoretical background}





















% ---------------------------------------------------------

\newpage

\section{Hipótesis}

Lorem ipsum dolor sit amet, consectetur adipiscing elit, sed do eiusmod tempor incididunt ut labore et dolore magna aliqua. Ut enim ad minim veniam, quis nostrud exercitation ullamco laboris nisi ut aliquip ex ea commodo consequat. Duis aute irure dolor in reprehenderit in voluptate velit esse cillum dolore eu fugiat nulla pariatur. Excepteur sint occaecat cupidatat non proident, sunt in culpa qui officia deserunt mollit anim id est laborum.


\chapter{Metodología}

Lorem ipsum dolor sit amet, consectetur adipiscing elit, sed do eiusmod tempor incididunt ut labore et dolore magna aliqua. Ut enim ad minim veniam, quis nostrud exercitation ullamco laboris nisi ut aliquip ex ea commodo consequat. Duis aute irure dolor in reprehenderit in voluptate velit esse cillum dolore eu fugiat nulla pariatur. Excepteur sint occaecat cupidatat non proident, sunt in culpa qui officia deserunt mollit anim id est laborum.

\textbf{Etapa 1: Nombre Etapa 1}

\begin{itemize}

\item Lorem ipsum dolor sit amet, consectetur adipiscing elit, sed do eiusmod tempor incididunt ut labore et dolore magna aliqua. Ut enim ad minim veniam, quis nostrud exercitation ullamco laboris nisi ut aliquip ex ea commodo consequat. Duis aute irure dolor in reprehenderit in voluptate velit esse cillum dolore eu fugiat nulla pariatur. Excepteur sint occaecat cupidatat non proident, sunt in culpa qui officia deserunt mollit anim id est laborum.

\end{itemize}

\textbf{Etapa 2: Nombre Etapa 2}

\begin{itemize}

\item Lorem ipsum dolor sit amet, consectetur adipiscing elit, sed do eiusmod tempor incididunt ut labore et dolore magna aliqua. Ut enim ad minim veniam, quis nostrud exercitation ullamco laboris nisi ut aliquip ex ea commodo consequat. Duis aute irure dolor in reprehenderit in voluptate velit esse cillum dolore eu fugiat nulla pariatur. Excepteur sint occaecat cupidatat non proident, sunt in culpa qui officia deserunt mollit anim id est laborum.

\end{itemize}


\chapter{Desarrollo del proyecto}

\section{Seccion 1 Desarrollo del proyecto}

Lorem ipsum dolor sit amet, consectetur adipiscing elit, sed do eiusmod tempor incididunt ut labore et dolore magna aliqua. Ut enim ad minim veniam, quis nostrud exercitation ullamco laboris nisi ut aliquip ex ea commodo consequat. Duis aute irure dolor in reprehenderit in voluptate velit esse cillum dolore eu fugiat nulla pariatur. Excepteur sint occaecat cupidatat non proident, sunt in culpa qui officia deserunt mollit anim id est laborum.\\


\chapter{Discusión de resultados}

\section{Validación}

Lorem ipsum dolor sit amet, consectetur adipiscing elit, sed do eiusmod tempor incididunt ut labore et dolore magna aliqua. Ut enim ad minim veniam, quis nostrud exercitation ullamco laboris nisi ut aliquip ex ea commodo consequat. Duis aute irure dolor in reprehenderit in voluptate velit esse cillum dolore eu fugiat nulla pariatur. Excepteur sint occaecat cupidatat non proident, sunt in culpa qui officia deserunt mollit anim id est laborum.\\


\chapter{Conclusiones y consideraciones finales}

Lorem ipsum dolor sit amet, consectetur adipiscing elit, sed do eiusmod tempor incididunt ut labore et dolore magna aliqua. Ut enim ad minim veniam, quis nostrud exercitation ullamco laboris nisi ut aliquip ex ea commodo consequat. Duis aute irure dolor in reprehenderit in voluptate velit esse cillum dolore eu fugiat nulla pariatur. Excepteur sint occaecat cupidatat non proident, sunt in culpa qui officia deserunt mollit anim id est laborum.\\


\begin{algorithm}
    \caption{Nombre Algoritmo}\label{code: Filtro de partículas}
    \begin{algorithmic}[1]
        \Function{NombreDeFuncion}{$parameter1, parameter2$}
        \For{$i = 1$; $M$}
        \State Something ${\epsilon}_{k-1}(i) \sim N(0,{E}_{t})$ 
        \Comment{Comment something}
        \EndFor
        \EndFunction
    \end{algorithmic}
\end{algorithm}




\begin{figure}[H]
	\centering
	\begin{subfigure}{.5\textwidth}
		\centering
		\includegraphics[width=0.65\linewidth]{assets/imgs/logoeia.png}
		\caption{Pie de foto de imagen A, \cite{CITA1}}
		\label{Fig: figure1name}
	\end{subfigure}~
	\begin{subfigure}{.5\textwidth}
		\centering
		\includegraphics[width=0.65\linewidth]{assets/imgs/logoeia.png}
		\caption{Pie de foto de imagen B, \cite{CITA2}}
		\label{Fig: figure2name}
	\end{subfigure}%
	\caption{Pie de foto imagen general}
	\label{Fig: figurename}
\end{figure}


%------------------------------------------------------
% GENERATE BIBLIOGRAPHY BASED ON ALL CAPTIONS AND REFERENCES

\begin{sloppypar}
    \bibliography{references}
    \bibliographystyle{apalike}
\end{sloppypar}



\end{document}
